% Zeilenabstand ------------------------------------------------------------
\onehalfspacing


% Seitenränder -------------------------------------------------------------
% bindingoffset bei doppelseitigen Druck kann eine Bindekorrektur definiert werden
\geometry{left=25mm,right=35mm,top=13mm,bottom=15mm,includeheadfoot}

% Kopf- und Fußzeilen ------------------------------------------------------
\pagestyle{scrheadings}

% Kopf- und Fußzeile auch auf Kapitelanfangsseiten -------------------------
\renewcommand*{\chapterpagestyle}{scrheadings}

% Schriftform der Kopfzeile ------------------------------------------------
\renewcommand{\headfont}{\normalfont}

% Kopfzeile ----------------------------------------------------------------
\ihead{\large{\textsc{\titel}}\\	\small{\untertitel} \\[2ex] \textit{\headmark}}
\chead{}
\ohead{\includegraphics[height=1.7cm]{\logo}}
\setlength{\headheight}{21mm} % Höhe der Kopfzeile
%\setheadwidth[0pt]{textwithmarginpar} % Kopfzeile über den Text hinaus verbreitern
\KOMAoptions{headsepline=0.4pt} % dicke der Trennlinie unter Kopfzeile

% Fußzeile -----------------------------------------------------------------
\ifoot{\copyright\ \autor}
\cfoot{}
\ofoot{\pagemark}
\setlength{\footheight}{10mm} % Höhe der Fußzeile
%\setfootwidth[0pt]{textwithmarginpar} % Fußzeile über den Text hinaus verbreitern
\KOMAoptions{footsepline=0.4pt} % dicke der Trennlinie unter Fußzeile


% erzeugt ein wenig mehr Platz hinter einem Punkt --------------------------
%\frenchspacing 

% Schusterjungen und Hurenkinder vermeiden
\clubpenalty = 10000
\widowpenalty = 10000 
\displaywidowpenalty = 10000


% Quellcode-Ausgabe formatieren --------------------------------------------
\lstset{numbers=left, numberstyle=\tiny, numbersep=5pt, breaklines=true}
\lstset{emph={square}, emphstyle=\color{red}, emph={[2]root,base}, emphstyle={[2]\color{blue}}}

% Fußnoten fortlaufend durchnummerieren ------------------------------------
\counterwithout{footnote}{chapter}


% Code-Listings
\lstset{
language=C, % Programmiersprache
showstringspaces=false, % In Strings keine Backspace zeichen
basicstyle=\ttfamily\footnotesize, % Schriftgröße small
breakatwhitespace=true,breaklines=true,%
tabsize=4, % Tabulatorbreite
commentstyle=\color{red},% Kommentarfarbe
keywordstyle=\color{blue}\textbf, % Keywörterfabe
%backgroundcolor=\color{codebackround}, % Hintergrundarbe
keywordstyle=\color{blue}\textbf, % Stile der eig. Keywörter (gleicher stile, wie bei Standart-c keywords
morekeywords={u_int8, u_int16, u_int32} % eigene Keywörter
showtabs=true, % Tabuloren anzeigen (disabled)
%numbers=left, % Zeilennummerierung auf der linken Seite (disabled)
%numberstyle=\tiny, % Zeilennummerierung kleine Schriftgröße (disabled)
%numbersep=-7pt, % vertikale Position der Nummerierung
%tab=\rightarrowfill
}
